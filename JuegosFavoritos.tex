\documentclass[12pt,oneside]{book}
\usepackage{geometry}                		% See geometry.pdf to learn the layout options. There are lots.
\geometry{a4paper}                   			% ... or a4paper or a5paper or ... 
%\geometry{landscape}                		% Activate for for rotated page geometry
%\usepackage[parfill]{parskip}    		% Activate to begin paragraphs with an empty line rather than an indent
\usepackage{graphicx}				% Use pdf, png, jpg, or epsß with pdflatex; use eps in DVI mode
								% TeX will automatically convert eps --> pdf in pdflatex		
\usepackage{amssymb}

\usepackage[spanish]{babel}			% Permite que partes automáticas del documento aparezcan en castellano.
\usepackage[utf8]{inputenc}			% Permite escribir tildes y otros caracteres directamente en el .tex
\usepackage[T1]{fontenc}				% Asegura que el documento resultante use caracteres de una fuente apropiada.

\usepackage{hyperref}				% Permite poner urls y links dentro del documento

\title{Mi Juego Favorito}
\author{Javier Tibau}
%\date{}							% Activate to display a given date or no date

\begin{document}
\maketitle
\tableofcontents

\chapter{Introducción}
El libro a continuación es creado como una herramienta para el desarrollo de habilidades de edición colaborativa de documentos de texto plano. La herramienta que habilita dicha colaboración, en este taller, es Git pero podría ser reemplazada por otros sistemas de versionamiento.

\chapter{Los Juegos}

\include{juegos/Buscaminas}
\include{juegos/Dota2}
\include{juegos/Zelda}
\section{Twisted Metal: }

\begin{figure}[htbp]
\begin{center}
\includegraphics[width=.60\textwidth]{./imagenes/skyward.jpg}
\caption{Zelda: Skyward Sword}
\label{Zelda: Skyward Sword}
\end{center}
\end{figure}
Zelda: Skyward Sword\footnote{\url{http://zelda.com/skywardsword/}} es el primer Zelda en ofrecerte total control de la espada y escudo via wiimote + nunchuck. Fue publicado en Noviembre del 2011 y ha sido uno de los juegos de Zelda más controversiales por su único modo de jugarlo con controles de movimiento.

La premisa del juego es que eres un chico (Link) que vive en una comunidad en una isla flotante y debes emprender un viaje a la superficie (Hyrule) para rescatar a Zelda.

\subsubsection{¿Por qué es uno de mis juegos favoritos?}
\begin{itemize}
\item[Gianni Carlo] Las reglas del juego son similares a los anteriores juegos de Zelda con el agregado que ahora cada enemigo fue diseñado con los controles de moviemiento en mente, no basta con agitar de izquierda a derecha el control para poder pasar el juego, y esto ayuda en gran parte a sumergirte en el juego ya que debes atacar de cierta forma a los enemigos y/o repeler ataques con tu escudo en el momento preciso sino recibes una penalidad. A pesar de esto, el juego no es atractivo para todo el mundo debido a la única opción de controles a los que alegan que no responden con suficiente precisión o a veces ni responden. Para mi, el gran interés del juego, aparte de presentar el origen de la historia del universo de Zelda, es el nuevo estilo de jugarlo y la experiencia de inmersión única que presenta a cualquier fan de la serie (los motion controls de Twilight Princess para Wii no cuentan en mi opinión ya que eran tan solo un test para ver que tan buena sería la acogida para implementarlo como única opción en el siguiente Zelda).
\end{itemize}


\chapter{Conclusiones}
Cuales juegos fueron más populares y un breve razonamiento del porqué.

\end{document}  
