\documentclass[12pt,oneside]{book}
\usepackage{geometry}                		% See geometry.pdf to learn the layout options. There are lots.
\geometry{a4paper}                   			% ... or a4paper or a5paper or ... 
%\geometry{landscape}                		% Activate for for rotated page geometry
%\usepackage[parfill]{parskip}    		% Activate to begin paragraphs with an empty line rather than an indent
\usepackage{graphicx}				% Use pdf, png, jpg, or epsß with pdflatex; use eps in DVI mode
								% TeX will automatically convert eps --> pdf in pdflatex		
\usepackage{amssymb}

\usepackage[spanish]{babel}			% Permite que partes automáticas del documento aparezcan en castellano.
\usepackage[utf8]{inputenc}			% Permite escribir tildes y otros caracteres directamente en el .tex
\usepackage[T1]{fontenc}				% Asegura que el documento resultante use caracteres de una fuente apropiada.

\usepackage{hyperref}				% Permite poner urls y links dentro del documento

\title{Mi Juego Favorito}
\author{Javier Tibau}
%\date{}							% Activate to display a given date or no date

\begin{document}
\maketitle
\tableofcontents

\chapter{Introducción}
El libro a continuación es creado como una herramienta para el desarrollo de habilidades de edición colaborativa de documentos de texto plano. La herramienta que habilita dicha colaboración, en este taller, es Git pero podría ser reemplazada por otros sistemas de versionamiento.

\chapter{Los Juegos}

\include{juegos/Buscaminas}
\include{juegos/Dota2}
\include{juegos/Zelda}
\section{Top Gear}

\begin{figure}[htbp]
\begin{center}
\includegraphics[width=.60\textwidth]{./imagenes/top_gear.jpg}
\caption{Top Gear}
\label{Top Gear}
\end{center}
\end{figure}
Top Gear\footnote{\url{http://es.wikipedia.org/wiki/Top_Gear_%28videojuego%29}} es un videojuego de carreras para Super Nintendo publicado en el año 1992. En el juego tienes que competir con 20 automóviles para llegar en el primer lugar y avanzar hacia el siguiente torneo. El modo de juego es sencillo, tan solo debes debes indicar mediante teclas la direción a la cual deseas que se dirija el auto.

\subsubsection{¿Por qué es uno de mis juegos favoritos?}
\begin{itemize}
\item[Saulo Ronquillo]No es un juego muy complejo ni goza de gráficas espectaculares pero hasta el día de hoy me sigue entreteniendo mucho, durante el transcurso del juego las competencias se realizan en diferentes lugares del mundo lo cual ayuda a que el juego no sea monótomo y agregado a todo eso, la música\footnote{\url{http://archive.org/details/TopGearMusicaSnes}} del videojuego es muy, muy buena.
\end{itemize}


\chapter{Conclusiones}
Cuales juegos fueron más populares y un breve razonamiento del porqué.

\end{document}  
