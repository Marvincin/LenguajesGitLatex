\section {God Of War III}

\begin{figure}[htbp]
\begin{center}
%\includegraphics[width=.60\textwidth]{./imagenes/gow3.jpg}
\caption{God Of War III}
\label{God Of War III}
\end{center}
\end{figure}
En la guerra entre Titanes y Dioses, varios de los primeros mueren en combate, pero Kratos con ayuda de Gaia asesina a Poseidón, lo que causa una inundación que elimina a los sirvientes del Olimpo. Pero cuando se encuentra con Zeus él arroja al guerrero y a Gaia a un abismo, la Titanide traiciona a Kratos para poder salvarse y deja que este caiga en el río de las almas, que lo debilita. A la salida de dicho río se encuentra con el fantasma de Atenea quién le ofrece sus consejos, las Espadas del Exilio y le exige asesinar a Zeus, pues afirma que su estado actual le ha permitido ver la realidad de las cosas. En su camino se encuentra con una estatua que lo llama "padre". Él, esperanzado con que sea Calíope, su hija, se apresura a responder, para entender que era otra niña, Pandora, lo que descubre luego. Tiene un encuentro con Hades y lo asesina consiguiendo las preciadas Garras de Hades, provocando el descontrol sobre la muerte, por lo que los mortales mueren descontroladamente a causa de una peste. Luego de la matanza del Dios del Inframundo, Kratos se encuentra con Hefestos, quién le da ciertos consejos y le revela verdades sobre su pasado y el de su hija, Pandora. Kratos sigue su camino hacia Zeus, llega a la ciudad de Olimpia donde encuentra a Gaia apenas sostenida. No escucha los pedidos de ayuda de la Titanesa y le corta la mano para arrojarla al vacío. Ya en la ciudad, acaba con Helios para obtener su cabeza provocando a aún más caos por la ausencia del Sol. Luego el guerrero espartano se cruza con Hermes, a quién mata para poder robarle sus botas. Se enfrenta y asesina también a su hermanastro Hércules, de quien obtiene las Cestas de Nemea. Llega a las Fósas de Tártaro y se enfrenta en una cruenta lucha al Titán Cronos, quien cree haber acabado con el espartano, pero este ocasiona un agujero en su estómago que lo debilita, para luego clavar la espada del Olimpo en su frente. Después se encuentra en los jardines superiores. Allí tiene una discusión con Hera, a quien termina matando por burlarse de Pandora. Kratos supera el Laberinto de Dédalo y destruye a los tres jueces para poder acceder a la Llama del Olimpo. Al llegar a ella, Pandora se intenta sacrificar pero Kratos la detiene, cuando ella lo hace entrar en razón y se decide a dirigirse a la Llama del Olimpo para poder causar su efecto, aparece Zeus, quién pelea con Kratos (cuya distracción provoca que Pandora logre su cometido) en tres ocasiones. En el final de la segunda batalla, aparece la Titanesa Gaia quién intenta acabar con ambos, pero tanto Kratos como Zeus logran escapar al interior de Gaia donde tiene lugar la tercera y última batalla. Kratos asesina a Gaia y cree asesinar al Padre de los Dioses, que luego emerge de los escombros para matar al espartano. Zeus envía a Kratos a su mente oscura donde se presentan los viejos y malignos recuerdos de Kratos, pero él vuelve a la realidad con ayuda de Pandora y asesina violentamente a Zeus. Finalizada su venganza aparece Atenea explicándole que la maldad que había liberado de la caja de Pandora contaminó a los dioses causando el miedo y el odio que se generaba en su interior. Y que el verdadero poder que poseía era el de la esperanza, que ella había depositado en la caja para coexistir con el miedo. Exigiéndole Atenea los poderes que el había usado para matar a Zeus, Kratos se niega y desenfunda la Espada del Olimpo, Atenea se queja al espartano de querer asesinarla, pero Kratos no mata a la Diosa de Atenas, sino que atraviesa su propio estómago, liberando esperanza al mundo que parecía devastado. Al final de los créditos, aparecen rastros de sangre de Kratos, que dejan de ser visibles cuando se muestra un precipicio, donde posiblemente haya caído el Dios de la Guerra.

\subsubsection{¿Por qué es uno de mis juegos favoritos?}
\begin{itemize}
\item[Kevin Zambrano] Me gusta mucho este juego porque siempre me ha interesado todo lo referente a la mitología griega y este juego, aunque crea su propia historia, se asemeja mucho hacia la historial verdadera. El juego también es una secuela del God Of War II manteniendo su historia, conflictos, aventuras y demás hasta el final de su saga haceéndolo aún más intrigante. Aparte, este juego fue desarrollado únicamente para la consola de PlayStation 3 desarrollando unas gráficas realmente excelentes y un sonido de alta fidelidad desarrollando un ambiente de jugabilidad asombroso.
\end{itemize}
